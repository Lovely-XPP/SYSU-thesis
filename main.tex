\documentclass{sysuthesis} % 默认使用电子版(不填充空白页)。如果需要双面打印版,请注释掉本行并启用下一行
% \documentclass[print-both-sides]{sysuthesis} % 使用双面打印版(填充额外空白页以保证每一章开头都在奇数页)
\usepackage{sysucode}  % 在论文中使用代码

\input{docs/info}     % 论文相关信息
\input{docs/abstract}     % 摘要内容


\begin{document}
% 论文前置部分
\frontmatter
\pagenumbering{Roman}
\makeUndergraduateCover    % 封面
\makeUndergraduateTitlePage    % 扉页
\makedisclaim       % 学术诚信声明
\makeabstract       % 中英文摘要
\maketableofcontents        % 目录
\makelistoffiguretable      % 图表目录
\makenomenclature           % 主要符号表
\makenomenclaturetable      % 主要符号表
% example: \nomenclature{$\lambda$}{地理经度}

% 论文主体部分
\mainmatter

% 正文
\include{docs/chap01}
\newclearpage
\include{docs/chap02}
\newclearpage
\include{docs/chap03}
\newclearpage
\include{docs/chap04}
\newclearpage
\include{docs/chap05}
\newclearpage
\include{docs/chap06}
\newclearpage

% 结语

% 附录部分
\backmatter
% 参考文献. 因不需要纳入章节目录, 故放入附录部分
% 实际上参考文献是属于论文主体部分
\makereferences

% 附录
{
\appendix
\include{docs/appendix1}
\newclearpage
}

\include{docs/ack}    % 致谢
\newclearpage


\end{document}

